\chapter{Instalasi dan Konfigurasi Hive}\appcaption{Instalasi dan Konfigurasi Hive}

Langkah-langkah pemasangan dan konfigurasi Hive akan dijelaskan sebagai berikut,

\begin{enumerate}
  \item Pemasangan Hive
  \begin{enumerate}
    \item Pastikan perangkat lunak prasyarat sudah berhasil dipasang dan dilakukan konfigurasi. Sebelum dilakukan pemasangan Hive, diperlukan untuk mengunduh berkas Hive terlebih dahulu dengan perintah \verb|cd /usr/local|, dilanjutkan dengan \verb|sudo wget https://dlcdn.apache.org/hive/hive-2.3.9/|
    \newline \verb|apache-hive-2.3.9-bin.tar.gz|.
    \item Ekstrak berkas Hive yang sudah diunduh tadi dengan perintah \verb|sudo tar -xzvf apache-hive-2.3.9-bin.tar.gz|. Hasil ekstrak berkas Hive akan disimpan pada direktori yang sama.
    \item Selanjutnya, untuk memudahkan kedepannya, ganti nama folder Hive dengan perintah \verb|sudo mv apache-hive-2.3.9-bin hive|.
  \end{enumerate}
  \item Menambahkan Hive pada \textit{Environments Variables}
  \begin{enumerate}
    \item Hive perlu ditambahkan pada \textit{Environments Variabels} untuk memudahkan dalam melakukan eksekusi. Untuk menambahkannya, jalankan perintah \verb|sudo nano ~/.bashrc|.
    \item Tambahkan beberapa baris kode berikut pada akhir berkas \textit{bashrc}.
      \begin{lstlisting}[language=bash]
		# HIVE ENVIRONMENT
		export PATH=$PATH:/usr/local/hive/bin
      \end{lstlisting}
    \item Untuk mendapatkan perubahan dapat dilakukan dengan perintah \verb|source ~/.bashrc|.
  \end{enumerate}
\end{enumerate}