\chapter{HASIL DAN PEMBAHASAN}
Data return yang telah dipilih berdasarkan plot stasioneritasnya akan dilakukan pemodelan otoregresif. Dalam hal ini menggunakan tahap Box-Kenkins yang terdiri dari 3 tahapan umum sebagai berikut :
\section{Identifikasi Model}
%\usepackage{cleverref}
Identifikasi model dilakukan untuk menduga data yang ada agar dapat dimodelkan dengan otoregresif dalam tugas akhir ini. Pendugaan dapat dilakukan dengan melihat plot deret waktunya dan plot ACF serta PACF nya yang disesuaikan dengan hasil simulasi pada BAB II. \cite{
\section{Estimasi Paremeter}
Setelah diperoleh dugaan modelyaitu AR(1) dan AR(2), tahap yang sangat penting selanjutnya yaitu estimasi parameter untuk mendapatkan persamaan model tersebut. Berikut ini hasil output estimasi dengan menggunakan perangkat lunak R pada model AR(1) dan AR(2), sebagai berikut:
%\nomenclature{\(AR_1\)}{Istilah pada Statistika}
%\nomenclature{\(\int_{pi}^{2*\pi}{f(x)}{dx}\)}{integral dari suatu fungsi}