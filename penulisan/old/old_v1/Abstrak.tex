
%\chapter{\textbf{ABSTRAK}}
%\begin{center}
%\textbf{{\LARGE{\textsc{MODEL PREDIKSI INAR(1) \\UNTUK STATUS KESEHATAN}}}}
%\end{center}
%\vspace{.8cm}
%\begin{center}
\addcontentsline{toc}{chapter}{ABSTRAK}
%\end{center}
\begin{center}
\textbf{\fontsize{12pt}{0}\selectfont{Judul Skripsi Anda}} \\
\vspace{14pt}
\textbf{\fontsize{12pt}{0}\selectfont{Nama Mahasiswa (1xx45xxxx)}}\\
\textbf{\fontsize{12pt}{0}\selectfont{Nama Pembimbing 1 beserta gelar}}\\
\textbf{\fontsize{12pt}{0}\selectfont{Nama Pembimbing 2 beserta gelar}}\\
\vspace{36pt}
\textbf{\fontsize{12pt}{0}\selectfont{ABSTRAK}}
\end{center}
\vspace{20pt}
\singlespacing
Abstrak yang dimaksudkan adalah ringkasan atau intisari, maksimum 200 kata atau satu halaman. Abstrak merupakan ikhtisar suatu tugas akhir yang memuat permasalahan, tujuan, metode penelitian, hasil, dan kesimpulan. Abstrak dibuat untuk memudahkan pembaca mengerti secara cepat isi tugas akhir untuk memutuskan apakah perlu membaca lebih lanjut atau tidak. Abstrak tidak memuat gambar maupun tabel, ditulis dengan huruf Times New Roman, 12 pt, satu spasi. Abstrak ditulis dalam bahasa Indonesia dan bahasa Inggris, masing-masing dimulai pada halaman baru. Abstrak hendaknya memuat satu kalimat yang menjelaskan latar belakang masalah, satu kalimat yang menjelaskan tujuan dan satu kalimat yang menjelaskan manfaat, satu kalimat yang menjelaskan lingkup dan satu kalimat yang menjelaskan batasan masalah, satu kalimat yang menjelaskan metodologi, percobaan maupun satu kalimat yang menjelaskan interpretasi data sertasatu kalimat yang menjelaskan hasil-hasil penelitian yang diperoleh. Di bawah abstrak, setelah satu baris kosong, tuliskan "Kata kunci:" diikuti lima kata kunci yang sesuai.


\singlespacing
\noindent 
\vspace{1ex}
\noindent
\begin{tabularx}{\textwidth}{@{}lX@{}}
    {\textbf{Kata kunci:}} & {Kata Kunci 1, Kata Kunci 2, Maksimal 5 Kata Kunci}
\end{tabularx}
