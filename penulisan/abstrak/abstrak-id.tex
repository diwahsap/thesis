\begin{abstractind}
\justifying

Perkembangan teknologi informasi mendorong peningkatan volume data yang dihasilkan dan disimpan setiap harinya. Hal ini menuntut \textit{platform} komputasi terdistribusi yang efisien dan \textit{scalable} untuk memproses data dalam skala besar. Hadoop dan Spark merupakan dua \textit{platform} populer yang menawarkan solusi untuk \textit{Big Data}. Penelitian ini bertujuan untuk membandingkan kinerja Hadoop dan Spark dalam mengolah data besar pada \textit{platform cloud} DigitalOcean dengan fokus pada beban kerja \textit{word count} dan \textit{sort}, yang merupakan dasar bagi banyak aplikasi \textit{data science}. \textit{Word count} digunakan dalam pembuatan \textit{Bag-of-Words} (BoW) untuk pemrosesan teks, sedangkan \textit{sort} penting dalam proses pembobotan TF-IDF. Kedua \textit{platform} diuji menggunakan \textit{benchmark} HiBench dengan variasi ukuran data mulai dari 100 KB hingga 15 GB. Hasil penelitian menunjukkan Spark mampu menyelesaikan tugas \textit{sort} dan \textit{word count} dengan waktu eksekusi yang jauh lebih cepat, khususnya pada data berukuran besar. Pada beban kerja \textit{sort}, Spark unggul mulai dari ukuran data 5 GB. Pada beban kerja \textit{word count}, Spark unggul mulai dari ukuran data 500 MB. Secara keseluruhan, Spark menunjukkan kinerja yang lebih baik dalam menangani data berukuran besar, sementara Hadoop lebih efisien untuk data berukuran kecil hingga menengah. Spark juga lebih efisien dalam memanfaatkan CPU dan memori, serta meminimalkan operasi \textit{disk} I/O. Hal ini menjadikan Spark \textit{platform} yang lebih \textit{scalable} dan efisien untuk pemrosesan data besar dibandingkan Hadoop, terutama untuk tugas \textit{word count} dan \textit{sort} yang menjadi fondasi bagi banyak aplikasi \textit{data science}. Temuan ini diharapkan dapat memberikan panduan bagi para praktisi dalam memilih \textit{platform} yang tepat untuk kebutuhan pemrosesan data.
%%pada abstrak bahasa Inggris, separator desimal koma

\bigskip
\noindent
\textbf{Kata kunci:} \textit{Big data}, Hadoop, HiBench, Komputasi awan, Pemrosesan terdistribusi paralel, \textit{Pseudo distributed}, Spark % masukkan keyword
\end{abstractind}
