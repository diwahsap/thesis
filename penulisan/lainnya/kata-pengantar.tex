%-----------------------------------------------------------------
%Di sini awal masukan untuk Prakata
%-----------------------------------------------------------------
\preface
\justifying

Puji syukur penulis ucapkan ke hadirat Allah SWT atas berkat dan rahmat-Nya sehingga skripsi ini dapat terselesaikan dengan baik.
Skripsi ini merupakan karya yang wajib dibuat oleh mahasiswa untuk menyelesaikan pendidikan sarjana di Institut Teknologi Sumatera. Penyusunan skripsi ini banyak mendapat bantuan dan dukungan dari berbagai pihak sehingga dalam kesempatan ini, dengan penuh kerendahan hati, penulis mengucapkan terima kasih kepada:

\begin{enumerate}
\item{Ibu Siti Ervingati dan bapak Kustriyanto, yang selalu memberikan doa, semangat, dukungan, dan motivasi sehingga penulis dapat mencapai tahap ini. Tak lupa pula untuk Dika, Habib, dan Syifa yang selalu bertanya, "Sudah makan, mas?".}
\item{Bapak Tirta Setiawan, S.Pd., M.Si., selaku Koordinator Program Studi Sains Data Fakultas Sains Institut Teknologi Sumatera dan Dosen Pembimbing Utama.}
\item{Bapak Riksa Meidy Karim, S.Kom., M.Si., M.Sc., dan Ibu Amalya Citra S.Kom., M.Si., M.Sc., selaku dosen pembimbing pendamping yang telah memberikan arahan, ilmu, motivasi, serta saran kepada penulis.}
\item{Seluruh dosen dan tenaga kependidikan Sains Data Institut Teknologi Sumatera yang telah memberikan banyak bantuan dan ilmu selama penulis berkuliah.}
\item{Abil, Imam, Sakul, dan sahabat-sahabat yang tidak dapat disebutkan satu persatu. Terima kasih atas semangat, bantuan dan motivasinya. Semoga kalian selalu dikuatkan.}
\item{Alfianri Manihuruk, teman-teman seperbimbingan, serta angkatan 2020 Sains Data Institut Teknologi Sumatera.}
\end{enumerate}

Penulis menyadari bahwa masih terdapat banyak kekurangan pada penulisan skripsi ini. Oleh karena itu, penulis mengharapkan kritik dan saran yang membangun dari pembaca demi perbaikan laporan ini. Semoga karya ini dapat bermanfaat bagi para pembaca pada umumnya dan juga bagi penulis pada khususnya.

\vspace{0.5cm}

\begin{flushright}
\begin{tabular}{p{7.5cm}l}
&Lampung Selatan, \approvaldatenc \\[2.5cm]
&\fullnamenc
\end{tabular}
\end{flushright}
