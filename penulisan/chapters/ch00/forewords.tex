\clearpage

\normalsize \bfseries \centering \MakeUppercase{Kata Pengantar}
\phantomsection% 
\addcontentsline{toc}{chapter}{Kata Pengantar}
\thispagestyle{fancy}
\fancyhf{}
\fancyhead[R]{\thepage}
\\[2\baselineskip]

\normalsize \normalfont \justifying
(Tuliskan maksud penulisan laporan, misal “Laporan penelitian ini dimaksud kan untuk memenuhi salah ”.........Pada halaman ini mahasiswa berkesempatan untuk menyatakan terima kasih secara tertulis kepada pembimbing dan pihak lain yang telah memberi bimbingan, nasihat, saran dan kritik, kepada mereka yang telah membantu melakukan penelitian, kepada perorangan atau lembaga yang telah memberi bantuan keuangan, materi dan/atau sarana.

Cara menulis kata pengantar beraneka ragam, tetapi hendaknya menggunakan kalimat yang baku. Ucapan terima kasih agar dibuat tidak berlebihan dan dibatasi pada pihak yang terkait secara ilmiah (berhubungan dengan subjek/materi penelitian). 

\flushright{
	Tempat penyusunan TA, tgl-bln-thn\\
	Penulis,
	\\[5\baselineskip]
	\theauthor
}

\clearpage
