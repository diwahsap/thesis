\chapter{KESIMPULAN DAN SARAN}

\section{Kesimpulan}
Penelitian ini melakukan evaluasi komprehensif terhadap kinerja Hadoop dan Spark dalam konteks pemrosesan \textit{Big Data}. Dengan menggunakan beban kerja \textit{Micro Benchmarks}, penelitian ini berhasil mengukur dan membandingkan performa kedua platform tersebut dalam mode \textit{pseudo-distributed}. Hasil eksperimen menunjukkan perbedaan signifikan dalam hal efisiensi dan penggunaan sumber daya antara Hadoop dan Spark, dengan Spark menunjukkan keunggulan dalam sejumlah aspek. Analisis ini memberikan wawasan berharga bagi organisasi yang ingin memilih platform Big Data yang tepat, memastikan keputusan mereka didasarkan pada informasi yang akurat dan relevan.

\section{Saran}
Untuk penelitian selanjutnya, disarankan untuk menggali lebih dalam aspek keamanan dan administrasi dari kedua \textit{platform} tersebut. Penelitian yang lebih fokus pada pengukuran aspek skalabilitas dan ketersediaan Hadoop dan Spark dalam berbagai konfigurasi kluster juga akan bermanfaat. Selain itu, penelitian tentang integrasi teknologi baru seperti kontainer dan orkestrasi kluster dalam pemrosesan \textit{Big Data} dapat menjadi topik yang menjanjikan. Akhirnya, implementasi kasus penggunaan nyata dan studi komparatif dalam lingkungan produksi akan memberikan pemahaman yang lebih dalam tentang kinerja dan kegunaan kedua \textit{platform} ini dalam skenario dunia nyata.