\chapter{PENUTUP}

\section{Kesimpulan}
Penelitian ini mengevaluasi dan membandingkan kinerja dua \textit{platform} komputasi terdistribusi populer, Hadoop dan Spark, dalam mengolah data teks khususnya penerapan algoritma \textit{sort} dan \textit{word count} pada \textit{platform cloud} DigitalOcean. Berdasarkan hasil penelitian, Spark menunjukkan performa yang lebih unggul dibandingkan Hadoop dalam sebagian besar skenario. Spark mampu menyelesaikan tugas \textit{sort} dan \textit{word count} dengan waktu eksekusi yang lebih cepat dan \textit{throughput} yang lebih tinggi, terutama pada data berukuran besar (di atas 500 MB untuk \textit{word count} dan 5 GB untuk \textit{sort}). Hal ini disebabkan arsitektur \textit{in-memory} Spark yang memungkinkan pemrosesan data lebih cepat dengan meminimalkan akses ke \textit{disk}. Analisis penggunaan sumber daya menunjukkan bahwa Spark lebih efisien dalam memanfaatkan CPU dan memori, serta meminimalkan operasi \textit{disk I/O} dibandingkan Hadoop.  Hal ini berkontribusi pada performa dan skalabilitas Spark yang lebih baik dalam menangani data besar. 

Hasil penelitian ini menegaskan bahwa Spark merupakan pilihan yang lebih tepat untuk pemrosesan data besar dibandingkan Hadoop, terutama jika \textit{throughput} dan waktu eksekusi menjadi pertimbangan utama. 

\section{Saran}
Penelitian selanjutnya dapat memfokuskan dengan menguji performa Hadoop dan Spark pada beban kerja \textit{machine learning}, dan menganalisis pengaruh penambahan jumlah \textit{node} terhadap skalabilitas. Dengan menjalankan penelitian lanjutan yang menggabungkan aspek-aspek tersebut, pemahaman yang lebih komprehensif mengenai keunggulan dan kelemahan Hadoop dan Spark dapat diperoleh, sehingga memungkinkan pemilihan \textit{platform} yang optimal berdasarkan kebutuhan spesifik pemrosesan data.