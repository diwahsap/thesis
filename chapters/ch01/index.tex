\chapter{Pendahuluan}

\pagestyle{plain}

\section{Latar Belakang}

Pengolahan dan analisis data telah menjadi bagian penting dalam berbagai aspek kehidupan modern \cite{vermaBigDataManagement2016}. Organisasi, perusahaan, dan lembaga pemerintah mengandalkan data untuk pengambilan keputusan, inovasi produk, pengembangan layanan pelanggan, dan banyak aspek lainnya. Seiring dengan meningkatnya volume data yang dihasilkan setiap hari, tantangan utama yang dihadapi adalah bagaimana mengelola dan menganalisis data ini dengan efisien dan cepat \cite{ahmadvandGapproxUsingGallup2019}. Beberapa solusi telah tersedia untuk mengatasi masalah ini. Salah satu solusi yang paling efisien adalah komputasi terdistribusi. 

Komputasi terdistribusi adalah cara untuk mencapai paralelisme dengan menggabungkan beberapa mesin independen yang berbeda \cite{bhattacharyaEvaluatingDistributedComputing2021}. Dalam komputasi terdistribusi, data besar dibagi ke dalam sejumlah \textit{node} atau server yang bekerja bersama-sama untuk mengolahnya. Dua teknologi yang umum digunakan dalam komputasi terdistribusi ini adalah Apache Hadoop dan Apache Spark. Hadoop dan Spark adalah dua platform komputasi \textit{big data} yang paling populer dan banyak digunakan di seluruh dunia. \textit{Platform} ini menawarkan berbagai kemampuan untuk mengelola, menyimpan, dan menganalisis data dalam skala besar. 

MapReduce adalah alat yang digunakan untuk komputasi terdistribusi, dirancang khusus untuk menulis, membaca, dan memproses jumlah data yang besar \cite{deanMapReduceSimplifiedData2004}. Pemrosesan data dalam MapReduce ini terdiri dari tiga tahap: fase \textit{Map}, fase \textit{Shuffle}, dan fase \textit{Reduce}. Dalam teknik ini, berkas-berkas besar dibagi menjadi beberapa blok kecil dengan ukuran yang sama dan didistribusikan ke seluruh klaster untuk penyimpanan. MapReduce dan sistem file terdistribusi (HDFS) adalah bagian inti dari sistem Hadoop, sehingga komputasi dan penyimpanan bekerja bersama-sama di seluruh \textit{node} yang membentuk klaster komputer \cite{samadiComparativeStudyHadoop2016}. Hadoop MapReduce memerlukan akses ke penyimpanan untuk membaca dan menulis data, sehingga dapat memperlambat proses komputasi, sehingga hadirlah Spark.

Spark, di sisi lain, menawarkan teknologi \textit{Resilient Distributed Datasets} (RDDs) untuk mendukung proses \textit{Map} dan \textit{Reducing} secara lebih efektif dan cepat \cite{ahmadvandGapproxUsingGallup2019}. Spark bukan hanya alternatif Hadoop, tetapi juga menyediakan berbagai fungsi, misalnya mendukung \textit{MLib}, \textit{GraphX}, dan \textit{Spark streaming} untuk analisis data besar \cite{zahariaSparkClusterComputing2010}. Spark menggunakan memori untuk menyimpan data sehingga dapat mengurangi siklus baca dan tulis. Perbedaan mendasar ini mengakibatkan menarik untuk melihat perbandingan performa antara keduanya. Salah satu cara untuk membandingkan performa keduanya adalah menggunakan tolok ukur Hibench.

Tolok ukur HiBench adalah salah satu tolok ukur kinerja yang paling sering digunakan. HiBench mencakup sejumlah tugas \textit{benchmarking} yang mencerminkan berbagai jenis pemrosesan data, seperti pengolahan batch, aliran data, \textit{query}, atau pun \textit{machine learning} \cite{huangHiBenchBenchmarkSuitea}. Oleh karena itu, HiBench adalah alat yang cocok untuk mengukur dan membandingkan kinerja antara Hadoop dan Spark dalam berbagai skenario penggunaan.

Penelitian tentang evaluasi performa Hadoop dan Spark menggunakan HiBench telah beberapa kali dilakukan. Shi et al. \cite{shiClashTitansMapReduce2015} melakukan penelitian dengan dua alat yang dirancang untuk mengukur kinerja MapReduce dan Spark dalam berbagai skenario beban kerja. Penelitian ini mengevaluasi kinerja dalam pekerjaan \textit{batch} dan iteratif, dengan fokus pada komponen-komponen penting seperti \textit{shuffle}, dan \textit{caching}. Hasil penelitian mereka menunjukkan bahwa Spark lebih cepat daripada Hadoop dalam beberapa kasus, terutama ketika menangani tugas-tugas pemrosesan data yang lebih kecil. Namun, ketika ukuran data meningkat, Hadoop terbukti lebih efisien. Selanjutnya, perbandingan kinerja antara Hadoop dan Spark juga disorot oleh penelitian Samadi et al. \cite{samadiComparativeStudyHadoop2016}, yang menggunakan delapan tolok ukur dari HiBench. Penelitian ini menunjukkan bahwa Spark cenderung lebih efisien ketika menangani data dalam jumlah kecil atau saat memproses tugas dalam memori, sementara Hadoop lebih sukses ketika beban kerja melibatkan operasi I/O penyimpanan yang intensif. Selain itu, penelitian oleh Satish dan Rohan \cite{gopalaniComparingApacheSpark2015} menyoroti perbandingan kinerja antara Hadoop dan Spark khususnya dalam konteks algoritma \textit{K-means}. Penelitian itu menemukan bahwa Spark dapat mencapai kecepatan hingga tiga kali lipat dibandingkan Hadoop, dengan catatan bahwa performa Spark sangat bergantung pada ukuran memori yang memadai.

Berdasarkan penelitian sebelumnya, penelitian ini bertujuan untuk menyelidiki perbandingan kinerja antara Hadoop dan Spark dengan menggunakan tolok ukur HiBench dengan studi kasus tertentu. Dengan pemahaman mendalam mengenai kekuatan dan kelemahan masing-masing \textit{platform} dalam berbagai konteks pemrosesan data, organisasi atau peneliti dapat membuat keputusan yang lebih informasional saat memilih \textit{platform} yang paling sesuai dengan kebutuhan mereka. Selain itu, penelitian ini akan dilakukan dengan memanfaatkan Infrastruktur sebagai Layanan (IaaS) yang disediakan oleh \textit{DigitalOcean}, memungkinkan penggunaan sumber daya komputasi dalam skala yang fleksibel dan efisien. Dengan demikian, penelitian ini akan memberikan kontribusi berharga dalam membantu pemangku kepentingan dalam mengoptimalkan pemrosesan data dalam lingkungan komputasi terdistribusi.

\section{Rumusan Masalah}
Adapun rumusan masalah dalam penelitian ini adalah sebagai berikut:
\begin{enumerate}
	\item Bagaimana implementasi HiBench pada Hadoop dan Spark di DigitalOcean?
	\item Bagaimana kinerja Hadoop dan Spark ketika diuji menggunakan  beban kerja \textit{Micro Benchmarks} yang disediakan oleh HiBench?
	\item Bagaimana perbandingan kinerja antara Hadoop dan Spark dalam mode pseudo-distribusi dalam konteks pemrosesan data dalam skala besar dengan menggunakan tolok ukur HiBench?
\end{enumerate}

\section{Tujuan}
Penelitian ini memiliki tujuan, yaitu:
	\begin{enumerate}
		\item Untuk mengetahui implementasi HiBench pada Hadoop dan Spark di DigitalOcean
		\item Untuk mengukur kinerja Hadoop dan Spark ketika diuji menggunakan  beban kerja \textit{Micro Benchmarks} yang disediakan oleh HiBench
		\item Untuk membandingkan kinerja antara Hadoop dan Spark dalam mode pseudo-distribusi saat memproses data dalam skala besar dengan menggunakan tolok ukur HiBench
	\end{enumerate}

\section{Manfaat}
Hasil dari penelitian ini diharapkan akan memberikan manfaat sebagai berikut:
\begin{enumerate}
	\item 
	Penelitian ini akan memberikan informasi yang berguna bagi organisasi yang sedang mempertimbangkan pemilihan platform \textit{Big Data}, sehingga mereka dapat membuat keputusan yang lebih terinformasi.	
	\item
	Penelitian ini akan membantu dalam memahami lebih dalam kinerja Hadoop dan Spark dalam berbagai skenario pemrosesan data.
	\item
	Hasil dari penelitian ini dapat menjadi dasar untuk penelitian lebih lanjut dalam pengembangan dan peningkatan platform \textit{Big Data.}
\end{enumerate}

\section{Batasan Masalah}
Penelitian ini memiliki beberapa batasan yang perlu diperhatikan sebagai berikut:
	\begin{enumerate}
		\item 
		Penelitian ini akan fokus pada perbandingan kinerja antara Hadoop dan Spark dalam mode \textit{pseudo-distributed.}
		\item
		Pengujian kinerja akan menggunakan HiBench, sebuah tolok ukur kinerja yang umum digunakan dalam penelitian \textit{Big Data.}
		\item
		Implementasi Hadoop dalam lingkungan komputasi awan akan menggunakan salah satu penyedia awan, yaitu \textit{DigitalOcean}.
		\item
		Penelitian ini akan berfokus pada aspek kinerja. Aspek lain seperti keamanan dan administrasi tidak akan dibahas secara rinci.
	\end{enumerate}

%\section{Metodologi}
%
%Tuliskan semua tahapan yang akan dilalui selama pelaksanaan tugas akhir. Tahapan ini spesifik untuk menyelesaikan persoalan tugas akhir. Tahapan studi literatur tidak perlu dituliskan karena ini adalah pekerjaan yang harus Anda lakukan selama proses pelaksanaan tugas akhir.
%
%\section{Sistematika Pembahasan}
%
%Subbab ini berisi penjelasan ringkas isi per bab. Penjelasan ditulis satu paragraf per bab buku.
