\chapter{TINJAUAN PUSTAKA}
\section{Latar Belakang}
Pada bagian ini masalah pokok dan urgensi yang menjadi latar belakang penelitian disajikan. Bagian latar belakag hendaknya memuat: pokok permasalahan, manfaat penelitian (mengapa subjek/tema penelitian penting untuk dikaji) dan telaahan tentang penelitian yang telah dilakukan peneliti-peneliti sebelumnya secara ringkas \cite{muthiahPerformanceEvaluationHadoopa} \cite{shiClashTitansMapReduce2015}