\chapter{TINJAUAN PUSTAKA}
\section{Latar Belakang}
Pada bagian ini masalah pokok dan urgensi yang menjadi latar belakang penelitian disajikan. Bagian latar belakag hendaknya memuat: pokok permasalahan, manfaat penelitian (mengapa subjek/tema penelitian penting untuk dikaji) dan telaahan tentang penelitian yang telah dilakukan peneliti-peneliti sebelumnya secara ringkas \cite{muthiahPerformanceEvaluationHadoopa}

Telaahan terhadap berbagai penelitian terdahulu dapat disarikan dari Bab II Tinjauan Pustaka. Telaahan ini hendaknya bermuara pada sisi-sisi kajian atau pokok persoalan yang akan diteliti dalam penelitian ini

\section{Rumusan Masalah}
Tuliskan secara jelas, rumusan masalah dalam penelitian Anda.
\begin{enumerate}{\tiny }
	\item 
	Point pertama?
	\item
	Point dua?
	\item
	Point tiga?
\end{enumerate}
\section{Tujuan}
Tuliskan secara jelas, tujuan umum maupun tujuan khusus penelitian.
\begin{enumerate}
	\item 
	Point satu.
	\item
	Point dua.
	\item
	Point tiga.
\end{enumerate}
\section{Batasan Masalah}
Memuat batas-batas kajian dengan jelas termasuk asumsi-asumsi yang digunakan selama penelitian. Batas-batas penelitian dapat berupa komposisi bahan (diambil dari daerah tertentu atau spesies tertentu), umpan (apakah umpan sintetik atau nyata), alat (alat jenis tertentu) dan sebagainya.