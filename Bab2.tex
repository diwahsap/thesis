\pagestyle{fancy}
\fancyhead[L]{\textsl{Kajian Pustaka}}
\fancyhead[R]{}
\fancyfoot[C]{\thepage}
\chapter{KAJIAN PUSTAKA}\label{bab2}

Pada bab ini dikaji beberapa hal yang berkaitan dengan penelitian-penelitian sebelumnya, yaitu berkaitan dengan Klasifikasi Linier. Selain itu diberikan penjelasan mengenai permasalahan Klasifikasi Linier.
\section{Panduan Penulisan Laporan Penelitian}

\subsection{Kertas}
Laporan dicetak pada kertas HVS berukuran A4 (210 mm x 297 mm) dan berat 80 g/m2, satu kolom dengan batas 40 mm dari tepi kiri dan 30 mm atas kertas, 30 mm dari tepi kanan kertas dan 30 mm dari tepi bawah kertas. Mulai dari Bab I jika jumlah halaman laporan lebih dari 150 halaman maka harus dicetak BOLAK-BALIK guna menghemat konsumsi kertas agar lebih ramah lingkungan, diperbolehkan merubah berat kertas yang dipakai. Namun jika jumlah halaman laporan kurang dari 150 maka diperbolehkan tidak bolak-balik. Laporan dijilid menggunakan Hard Cover.

\subsection{Pencetakan}
\begin{itemize}
	\item
	Laporan dibuat dengan huruf jenis Times New Roman 12 pt, tinta hitam. Gambar-gambar dicetak berwarna. Tabel dapat menggunakan ukuran huruf 11 pt, untuk penulisan sumber tabel dan gambar menggunakan ukuran huruf 10 pt. 
	(b) Baris-baris kalimat berjarak satu setengah spasi dan rata kanan-kiri, kecuali untuk bagian-bagian yang diatur khusus seperti pada ketentuan butir c berikut. 
	(c) Penyimpangan terhadap jarak satu setengah spasi tersebut (menjadi satu spasi) dilakukan untuk abstrak, gambar/tabel, judul/keterangan gambar/tabel, daftar pustaka, daftar simbol dan lampiran-lampiran. 
	(d) Antara baris pertama paragraf baru dan baris terakhir paragraf yang mendahuluinya diberi satu baris kosong. Huruf pertama paragraf baru dimulai dari batas tepi kiri naskah (tidak menjorok ke dalam). 
	(f) Hard Cover laporan berwarna hitam untuk Tugas Akhir, dengan kertas pembatas berwarna merah  bergambar logo itera (di bagian tegah kertas) di per bagian bab. Pada bagian sisi samping luar hard cover diberi logo Itera, judul tugas akhir, nama dan nim mahasiswa. Untuk lembar pengesahan memakai kertas khusus kertas jeruk (120 gr). Laporan memiliki tali pembatas berwarna kuning gold. 
\end{itemize}

\subsection{Bahasa}
Laporan dapat ditulis dalam bahasa Indonesia atau bahasa Inggris. Penggunaan bahasa mengikuti ejaan yang baku dan konsisten. Sebagai contoh, dalam merujuk sebuah artikel ilmiah, gunakan singkatan "dkk." jika menggunakan bahasa Indonesia dan "et al." jika menggunakan bahasa Inggris.

\subsection{Penomoran Halaman}
Nomor halaman terdiri dari dua bagian: nomor kelompok dan nomor halaman, keduanya membentuk sebuah footer (dalam satu baris), 20 mm dari tepi bawah kertas. 
(a).	Nomor halaman bagian depan laporan ditulis dengan angka Romawi i, ii, iii, iv, ..., ... x, xi, ... Bagian tengah dan belakang laporan ditulis dengan angka Arab. 1, 2, 3, .... Nomor halaman bagian belakang laporan adalah kelanjutan dari nomor halaman bagian tengah laporan. 
(b).	Bagian Depan meliputi lembar Pengesahan, Abstrak, Kata Pengantar, Daftar Isi, Daftar Tabel, Daftar Gambar dan Daftar Simbol. Bagian Tengah meliputi Bab 1 s/d 5. Bagian Belakang meliputi Daftar Pustaka, dan Lampiran-lampiran. Tidak diperkenankan menuliskan header dan footer berupa nomor halaman. Penulisan nomor halaman terletak di tengah bawah setiap lembar kecuali cover paling depan dengan jenis  font times new roman ukuran 11 pt

\subsection{Gambar}
Gambar mencakup juga grafik, diagram, denah, peta, bagan, monogram, diagram alir, dan foto. Gambar diletakkan rata tengah (center) terhadap batas margin. Antara gambar dan baris kalimat yang ada sebelum maupun sesudah gambar harus diberi jarak paling sedikit saru baris kosong. Nomor dan judul gambar diletakan di bawah gambar. Judul gambar harus sama dengan judul gambar yang tercantum pada halaman daftar gambar. Setiap gambar diberi nomor. Nomor gambar terdiri atas dua angka Arab yang dipisahkan oleh sebuah titik. Angka pertama menunjukkan nomor bab tempat gambar tersebut dimuat, sedangkan angka kedua menunjukkan nomor urut gambar dalam bab. Judul gambar ditulis dengan huruf kecil, kecuali huruf pertama kata pertama yang ditulis dengan huruf kapital. Semua gambar pada laporan digambar ulang oleh penulis. Pencantuman sumber dilakukan jika gambar bukan merupakan hasil analisis. Pada gambar hasil analisis data tidak dicantumkan sumber. Gambar yang dibuat landscape, bagian bawah gambar harus mengarah ke luar laporan.

\subsection{Tabel}
Tabel diletakkan di tengah (center) terhadap batas kertas. Antara tabel dan baris kalimat yang ada sebelum maupun sesudah tabel harus diberi jarak paling sedikit satu baris kosong. Nomor dan judul tabel diletakkan di atas tabel. Judul tabel harus sama dengan judul tabel yang tercantum pada halaman daftar tabel. Setiap tabel diberi nomor. Nomor tabel terdiri atas dua angka Arab yang dipisahkan oleh sebuah titik. Angka pertama menunjukkan nomor bab tempat tabel tersebut dimuat, sedangkan angka kedua menunjukkan nomor urut tabel dalam bab. Judul tabel ditulis dengan huruf kecil, kecuali huruf pertama kata pertama yang ditulis dengan huruf kapital. Semua tabel pada laporan penelitian harus ditulis ulang. Pada tabel hasil analisis data tidak dicantumkan sumber. Tabel yang dibuat landscape, bagian bawah gambar harus mengarah ke luar laporan. Tabel yang tidak memungkinkan untuk dibuat satu halaman, harus disambung ke halaman berikutnya.

\subsection{Penulisan Rumus}
Setiap rumus diberi nomor yang dituliskan di antara dua tanda kurung. Nomor rumus terdiri atas dua angka Arab yang dipisahkan oleh sebuah titik. Angka pertama menunjukkan bab tempat rumus tersebut dimuat dan angka kedua menunjukkan nomor urut rumus dalam bab. Nomor rumus ditempatkan di tepi kanan. Penulisan rumus menggunakan equation 3.0

\subsection{Kutipan}
Rumus, kalimat, paragraf, atau inti pengertian yang dikutip dari salah satu pustaka dalam daftar pustaka cukup ditunjukkan dengan amgka yang merujuk pada daftar pustaka, nomor dibuat di antara dua tanda kurung [ ]. Contoh: ..............pemuda menawarkan masa depan [1]. 

\section{Aturan Penulisan Daftar Pustaka}

[Catatan: beri 2 spasi kosong di antara judul DAFTAR PUSTAKA dengan isinya]
Tulispustaka yang dijadikanrujukandengan format penulisan yang ditetapkan. Setiapartikel yang dirujuk ditulis dan diurut secara alfabetik, ditulis mengikuti format daftar pustaka IEEE. Format penulisan jenis, prosiding, paten, tesis/disertasi, dan buku teks dapat dilihat pada contoh-contoh di bawah ini.

\subsection{Buku}
[1] S. M. Hemmingsen, Soft Science. Saskatoon: University of Saskatchewan Press, 1997.
[2] A. Rezi and M. Allam, "Techniques in array processing by means of transformations," in Control and Dynamic Systems, Vol. 69, Multi dimensional Systems, C. T. Leondes, Ed. San Diego: Academic Press, 1995, pp. 133-180. 
[3] D. Sarunyagate, Ed., Lasers. New York: McGraw-Hill, 1996.

\subsection{Jurnal yang Terbit Secara Periodik}
[4] G. Liu, K. Y. Lee and H. F. Jordan, "TDM and TWDM de Bruijn networks and shufflenets for optical communications," IEEE Transactions on Computers, vol. 46, pp. 695-701, June 1997.
[5] J. R. Beveridge and E. M. Riseman, "How easy is matching 2D line models using local search?" IEEE Transactions on Pattern Analysis and Machine Intelligence, vol. 19, pp. 564-579, June 1997.

\subsection{Artikel dari Published Conference Proceedings}
[6] N. Osifchin and G. Vau, "Power considerations for the modernization of telecommunications in Central and Eastern European and former Soviet Union (CEE/FSU) countries," in Second International Telecommunications Energy Special Conference, 1997, pp. 9-16.
[7] S. Al Kuran, "The prospects for GaAs MESFET technology in dc-ac voltage conversion," in Proceedings of the Fourth Annual Portable Design Conference, 1997, pp. 137-142.

\subsection{Makalah yang Dipresentasikan pada Seminar tetapi Tidak Dipublikasikan}
[7] H. A. Nimr, "Defuzzification of the outputs of fuzzy controllers," presented at 5th International Conference on Fuzzy Systems, Cairo, Egypt, 1996.

LAPORAN (technical reports, internal reports, memoranda) 
[8] K. E. Elliott and C. M. Greene, "A local adaptive protocol," Argonne National Laboratory, Argonne, France, Tech. Rep. 916-1010-BB, 1997.

\subsection{Tesis atau Disertasi}
[9] H. Zhang, "Delay-insensitive networks," M.S. thesis, University of Waterloo, Waterloo, ON, Canada, 1997.



\subsection{Manual}
[10] Bell Telephone Laboratories Technical Staff, Transmission System for Communications, Bell Telephone Laboratories, 1995.

\subsection{Catatan Kelas (\textit{Class Notes})}
[10] "Signal integrity and interconnects for high-speed applications," class notes for ECE 497-JS, Department of Electrical and Computer Engineering, University of Illinois at Urbana-Champaign, Winter 1997.

\subsection{Komunikasi Pribadi} 
[11] T. I. Wein (private communication), 1997.

\subsection{Sumber dari Internet}
[12] Computational, Optical, and Discharge Physics Group, University of Illinois at Urbana-Champaign, "Hybrid plasma equipment model: Inductively coupled plasma reactive ion etching reactors," December 1995, http://uigelz.ece.uiuc.edu/Projects/HPEM-ICP/index.html.
[13] D. Poelman (dirkpoelman@rug.ac.be), "Re: Question on transformerless power supply," Usenet post to sci.electronics.design, July 4, 1997.

\subsection{Katalog}
[14] Catalog No. MWM-1, Microwave Components, M. W. Microwave Corp., Brooklyn, NY. 

\subsection{\textit{Application Notes}}
[15] Hewlett-Packard, Appl. Note 935, pp. 25-29.

\subsection{Patents}
[16] K. Kimura and A. Lipeles, "Fuzzy Controller Component," U. S. Patent 14,860,040, December 14, 1996.

\subsection{Catatan untuk Diperhatikan}
\begin{itemize}
	\item 
	Semua pustaka yang tercantum pada daftar pustaka harus benar-benar dirujuk dalam penulisan laporan dan ditulis secara konsisten.
	\item
	Nama-nama penulis dalam rujukan harus ditulis secara lengkap, tidak boleh hanya menuliskan penulis pertama yang diikuti dengan "dkk.". 
	\item
	Perhatikan penggunaan tanda baca, misal titik dan koma dalam format ini. 
	\item
	Penulisan dalam badan teks hendaknya menggunakan ketentuan sebagai berikut: 
	\begin{enumerate}
		\item
		Penulis tidak ditulis sebagai bagian dalam kalimat
		\item
		Referensi dibuat dengan menggunakan angka dalam kurung siku
		\item
		Pada daftar pustaka [1] S. M. Hemmingsen, Soft Science. Saskatoon: University of Saskatchewan Press, 1997. Contoh penulisan: "Metode operasi dinamik ini telah dikembangkan[1]." Penulis ditulis bukan sebagai bagian dalam kalimat, hanya ditulis angka yang mewakili referensi pada pustaka.
	\end{enumerate}
\end{itemize}





% TODOO: sampai sini
\begin{equation}
\begin{aligned}	
	\hat{x} &= A \hat{x} \oplus Bu\oplus L(\hat{y}\oplus y) \\
	&= A\hat{x}\oplus Bu \oplus LC\hat{x}\oplus LCx \\
	&=(A\oplus LC)\hat{x} \oplus Bu \oplus LCx \\
	&=(A\oplus LC)^* Bu \oplus (A\oplus LC)^* LCx \\
	&=(A\oplus LC)^* Bu \oplus (A\oplus LC)^* LC(A^*Bu\oplus A^*Rw)
\end{aligned} \label{persamaanobserver}
\end{equation}

Dengan menerapkan identitas $ (A\oplus LC)^* = A^* (LCA^*)^* $, maka persamaan \ref{persamaanobserver} dapat dituliskan ulang menjadi sebagai berikut,

\begin{equation}
	\begin{aligned}	
		\hat{x} &= A^* (LCA^*)^* Bu \oplus A^* (LCA^*)^* LCA^* Bu\oplus A^* (LCA^*)^* LCA^* Rw)
	\end{aligned} \label{persamaanobserver1}
\end{equation}

Kemudian dengan menggunakan identitas bahwa $ (LCA^*)^* LCA^* = (LCA^*)^+ $, maka persamaan \ref{persamaanobserver1} dapat ditulis ulang menjadi sebagai berikut,

\begin{equation}
	\begin{aligned}	
		\hat{x} &= A^* (LCA^*)^* Bu \oplus A^* (LCA^*)^+ Bu\oplus A^* (LCA^*)^+ Rw)
	\end{aligned} \label{persamaanobserver2}
\end{equation}

Karena $ (LCA^*)^* \succeq (LCA^*)^+ $, maka persamaan \ref{persamaanobserver2} dapat dituliskan sebagai berikut

\begin{equation}
	\begin{aligned}	
		\hat{x} &= A^* (LCA^*)^* Bu \oplus A^* (LCA^*)^+ Rw) \\
		&= (A\oplus LC)^* Bu \oplus (A\oplus LC)^* LCA^* Rw
	\end{aligned} \label{persamaanobserver3}
\end{equation}

Seperti yang dijelakan sebelumnya, tujuan yang diharapkan adalah untuk menghitung matriks observer terbesar $ L $, dinotasikan sebagai $ L_opt  $, sehingga vektor keadaan yang diestimasi $ \hat{x} $ sedekat mungkin dengan keadaan $ x $, dan $ \hat{x} \preceq x $. Dengan kata lain yaitu menghitung $ L $ terbesar yang memenuhi pertidaksamaan berikut, $ \forall u,w $:

\begin{equation}
	\begin{aligned}	
		(A\oplus LC)^* Bu \oplus (A\oplus LC)^* LCA^* Rw \preceq A^* Bu \oplus A^*Rw) 
	\end{aligned} \label{persamaanobserver4}
\end{equation}

atau ekuivalen dengan

\begin{equation}
\begin{aligned}	
	& (A\oplus LC)^* B \preceq A^* B \\
	& (A\oplus LC)^* LCA^* R \preceq A^*Rw
\end{aligned} \label{persamaanobserver5}
\end{equation}
secara bersamaan.

\begin{teorema}
	Matriks terbesar L sehingga $(A^*LC) B = A^*B$ diberikan oleh
	\begin{equation}
		L_1=(A^* B)
	\end{equation}
\end{teorema}

\begin{teorema}
		Matriks terbesar L sehingga $(A^*LC) B = A^*B$ diberikan oleh
	\begin{equation}
		L_1=(A^* B)
	\end{equation}
\end{teorema}
\begin{teorema}
		Matriks terbesar L sehingga $(A^*LC) B = A^*B$ diberikan oleh
	\begin{equation}
		L_1=(A^* B)
	\end{equation}
\end{teorema}

\begin{bukti}
		Matriks terbesar L sehingga $(A^*LC) B = A^*B$ diberikan oleh
	\begin{equation}
		L_1=(A^* B)
	\end{equation}
\end{bukti}

%\subsection{Observer-Based Controllers}
%=========================================

%\begin{figure}
%	\begin{center}
%		\includegraphics[scale=.5]{Gambar/observer.jpg} %masih kopi paste (perlu diganti)
%	\end{center}
%	\caption{Sistem Max-Plus dan Desain Observer} \label{observer}
%\end{figure}

ini adalah tulisan pada subsubbab


\section{Judul Subbab}