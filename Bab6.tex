\chapter{Algoritma Dimensi Sentroidal}
\pagestyle{fancy}
\fancyhead[L]{\textsl{BAB VI : Algoritma Dimensi Sentroidal}}
\fancyhead[R]{}
\fancyfoot[C]{\thepage}
\onehalfspacing

Pada bab ini akan dibahas algoritma untuk menentukan dimensi sentroidal suatu graf terhubung $G$ dengan memanfaatkan matriks ketetanggaan dari $G$.
\section{Keterhubungan Graf}
Misalkan $G$ suatu graf dengan matriks ketetanggaan $A=A(G)$. Kita akan mengidentifikasi keterhubungan $G$ melalui $A(G)$. Misalkan $V(G)=\left\{v_1,v_2,\cdots,v_n\right\}$
\begin{lemma}\cite{Bi01}
	Banyaknya jalan dengan panjang $l$ pada graf $G$, dari titik $v_i$ ke $v_j$, adalah entri pada posisi ke $(i,j)$ dari matriks $A^l$.
\end{lemma}

\begin{proposisi}
	Graf $G$ terhubung jika dan hanya jika matriks $B=I+A+A^2+\cdots+A^{n-1}$ tidak memuat entri $0$.
\end{proposisi}
\begin{pf}
	Misalkan $G$ graf terhubung. Perhatikan bahwa karena setiap entri di $A$ memiliki nilai minimum $0$, maka begitu juga dengan $A^p$, untuk setiap $p$ bilangan bulat positif. Jelas bahwa untuk entri pada posisi ke $(i,i)$ dari matriks $B$ tidak bernilai $0$, untuk $1\le i\le n$. Ambil sebarang dua titik $v_i$ dan $v_j$, untuk $1\le i\ne j\le n$. Karena $G$ graf terhubung, maka $d(v_i,v_j)=k$, dengan $1\le k\le n-1$. Artinya, entri pada posisi ke $(i,j)$ pada matriks $A^k$ tidak bernilai $0$, begitu juga pada matriks $B$.\\
	Misalkan matriks $B$ tidak memuat entri $0$. Ambil sebarang posisi pada matriks $B$, misalkan $i$ dan $j$, dengan $i\ne j$. Artinya, terdapat $k\le n-1$ sehingga  entri dari matriks $A^k$ pada posisi ke $(i,j)$ adalah $p>0$. Maka, terdapat jalan dari titik $v_i$ dan $v_j$ pada graf $G$. Karena $i$ dan $j$ sebarang, maka $G$ graf terhubung.
\end{pf}

Dari proposisi di atas, kita dapat mengecek keterhubungan suatu graf dari matriks ketetanggaannya. \textbf{Algoritma Keterhubungan Graf} akan melakukan proses \textit{input} sebarang matriks ketetanggan dari suatu graf, kemudian dicek keterhubungannya. Jika graf tidak terhubung, maka akan di\textit{input} matriks ketetanggan yang baru, kemudian dicek kembali keterhubungannya. Proses ini akan berhenti sampai diperoleh matriks ketetanggan dari graf yang terhubung.

\begin{algorithm}
	\label{alg:the_alg}
	\floatname{algorithm}{Algoritma}
	\caption{Algoritma Keterhubungan Graf}
	\begin{algorithmic}[1]
		\Procedure{AKG}{}
		\State $n\gets$ sebarang bilangan asli \Comment{input banyak titik}
		\State $k\gets true$
		\While {$k=true$}
		\For {$i\leftarrow 1,n$}\Comment{Matriks ketetanggan A}
		\State $A(i,i)\gets0$
		\For {$j\leftarrow i+1,n$}
		\If{$random[0,1]<p(G)$}\Comment{$p(G)$ peluang sisi}
		\State $a\gets1$
		\Else
		\State $a\gets0$
		\EndIf
		\State $A(i,j)\gets a$
		\State $A(j,i)\gets a$
		\EndFor
		\EndFor
		\State $B\gets0$\Comment{$B=I+\cdots+A^{n-1}$}
		\For{$i\gets1,n$}
		\State $B\gets B+A^{i-1}$
		\EndFor
		\For{$i\gets1,n$}
		\For{$j\gets1,n$}
		\If{$B(i,j)=0$}
		\State $k\gets true$
		\State \textbf{break}
		\Else
		\State $k\gets false$
		\EndIf
		\EndFor
		\If{$k=true$}
		\State \textbf{break}
		\EndIf
		\EndFor
		\EndWhile
		\EndProcedure
	\end{algorithmic}
\end{algorithm}

\section{Matriks Jarak}
Setelah diperoleh matriks ketetanggan $A$ dari graf $G$ yang terhubung, kita definisikan $D$ sebagai matriks jarak, dengan entri $D_{i,j}$ adalah jarak titik $v_i$ ke titik $v_j$ pada graf $G$, untuk $1\le i,j\le n$. Proposisi berikut akan memberikan matriks $D$ jika diketahui matriks ketetanggan $A=A(G)$.
\begin{teorema}
	Misalkan suatu bilangan asli $m<n$ sehingga untuk setiap $1\le l\le m-1$ berlaku $A^l_{(i,j)}=0$, dan $A^m_{(i,j)}>0$. Maka $d(v_i,v_j)=m$.
\end{teorema}
\begin{pf}
	Andaikan $d(v_i,v_j)=k<m$. Artinya terdapat jalan dari $v_i$ ke $v_j$ dengan panjang $k$. Maka $A^k_{(i,j)}>0$, untuk $k<m$. Kontradiksi.
\end{pf}

\textbf{Algoritma Matriks Jarak} akan memberikan matriks jarak $D$, dengan \textit{input} matriks ketetanggaan $A=A(G)$, dan banyaknya titik $|V(G)|=n$.
%\vspace{-0.3in}
\begin{algorithm}
	\floatname{algorithm}{Algoritma}
	\caption{Algoritma Matriks Jarak}
	\begin{algorithmic}[1]
		\Procedure {AMJ}{$A$,$n$}\Comment{Matriks Jarak $D$}
		\For{$i\gets1,n$}
		\State $D(i,i)\gets0$
		\For{$j\gets i+1,n$}
		\State $t\gets1$
		\While{$t<n$}
		\State $C\gets A^t$
		\If {$C(i,j)>0$}
		\State $D(i,j)\gets t$
		\State $D(j,i)\gets t$
		\State \textbf{break}
		\Else
		\State $t\gets t+1$
		\EndIf
		\EndWhile
		\EndFor
		\EndFor
		\EndProcedure
	\end{algorithmic}
\end{algorithm}
%\vspace{-0.8in}
$ $\\
\section{Basis Sentroidal}
Dalam mencari basis sentroidal suatu graf terhubung $G$, salah satu cara yang mudah dilakukan adalah dengan mengumpulkan semua subhimpunan dari $V(G)$, dimulai dari subhimpunan dengan kardinalitas terkecil, kemudian diperiksa urutan parsial setiap titik $V(G)$ terhadap jaraknya dengan subhimpunan tersebut. Jika terdapat dua titik dengan urutan parsial yang sama, maka kita akan memproses kembali dengan subhimpunan yang baru.\\
Namun, ada beberapa hal yang perlu diperhatikan. Lemma \ref{LE01} memberikan kita titik mana saja yang harus termuat di dalam himpunan lokasi sentroidal. Ini jelas mempermudah kita dalam melakukan proses pencarian basis sentroidal. 
\vspace{-\topsep}
\begin{itemize}
	\item \textbf{[Algoritma Derajat Satu]} Titik berderajat satu pada $G$ haruslah termuat dalam himpunan lokasi sentroidal. Pada matriks $A$, kita mengidentifikasi baris yang hanya memuat satu entri bernilai 1.
	%\vspace{-0.4in}
	\begin{algorithm}
		\floatname{algorithm}{Algoritma}
		\caption{Algoritma Derajat Satu}
		\begin{algorithmic}[1]
			\Procedure{ADS}{$A$,$n$}
			\State $B\gets[\;\;]$\Comment{Himpunan Lokasi Sentroidal $B$}
			\State $l\gets1$
			\For {$j\gets1,n$}
			\If {Jumlah entri pada baris ke-$j=1$}\Comment{Pada Matriks $A$} 
			\State $B(l)\gets v_j$
			\State $l\gets l+1$	
			\EndIf
			\EndFor
			\EndProcedure
		\end{algorithmic}
	\end{algorithm}
	%\vspace{-0.5in}
	$ $\\
	Setelah mengidentifikasi titik berderajat satu, definisikan $Y=V(G)\backslash B$. Himpunan $Y$ ini yang kemudian kita periksa apakah terdapat dua titik yang memiliki himpunan \textit{open-neighborhood} yang sama.
	\item \textbf{[Algoritma \textit{open-neighborhood}]} Untuk dua titik yang memiliki himpunan \textit{open-neighborhood} yang sama, maka setidaknya satu dari dua titik tersebut termuat dalam himpunan lokasi sentroidal. Pada matriks $A$, kita mengidentifikasi dua baris dengan entri yang sama untuk setiap posisinya.
	%\vspace{-0.4in}
	\begin{algorithm}
		\floatname{algorithm}{Algoritma}
		\caption{Algoritma \textit{open-neighborhood}}
		\begin{algorithmic}[1]
			\Procedure{AON}{$A$,$B$,$Y$,$n$,$l$}
			\While{$1<|Y|$}
			\For{$k\gets2,|Y|$}
			\If{Entri baris ke indeks $Y(1)$ = Entri baris ke indeks $Y(k)$}\Comment{matriks $A$}
			\State $B(l)\gets Y(1)$
			\State $l\gets l+1$
			\State \textbf{break}
			\EndIf
			\EndFor
			\State $Y\gets Y\backslash\{Y(1)\}$
			\EndWhile
			\EndProcedure
		\end{algorithmic}
	\end{algorithm}
	%\vspace{-0.5in}
	$ $\\
	Setelah mengidentifikasi dua titik dengan himpunan \textit{open-neieghborhood} yang sama, definisikan $W=V(G)\backslash B$. Himpunan $W$ ini yang kemudian kita periksa apakah terdapat dua titik yang memiliki himpunan \textit{close-neighborhood} yang sama.
	\item \textbf{[Algoritma \textit{close-neighborhood}]} Untuk dua titik yang memiliki himpunan \textit{close-neighborhood} yang sama, maka setidaknya satu dari dua titik tersebut termuat dalam himpunan lokasi sentroidal. Pada matriks $X=A+I$, kita mengidentifikasi dua baris dengan entri yang sama untuk setiap posisinya.
	%\vspace{-0.6in}
	\begin{algorithm}
		\floatname{algorithm}{Algoritma}
		\caption{Algoritma \textit{close-neighborhood}}
		\begin{algorithmic}[1]
			\Procedure{ACN}{$A$,$B$,$W$,$n$,$l$}
			\State $X\gets A+I$
			\While{$1<|W|$}
			\For{$k\gets2,|W|$}
			\If{Entri baris ke indeks $W(1)$ = Entri baris ke indeks $W(k)$}\Comment{matriks $X$}
			\State $B(l)\gets W(1)$
			\State $l\gets l+1$
			\State \textbf{break}
			\EndIf
			\EndFor
			\State $W\gets W\backslash\{W(1)\}$
			\EndWhile
			\EndProcedure
		\end{algorithmic}
	\end{algorithm}
	%\vspace{-0.5in}
\end{itemize}
\vspace{-\topsep}
Setelah melalui ketiga algoritma di atas, diperoleh $B$ dan himpunan $Z=V(G)\backslash B$. Untuk selanjutnya, ada beberapa kasus : 
\vspace{-\topsep}
\begin{enumerate}
	\item Himpunan $B$ merupakan himpunan kosong, yaitu graf $G$ tidak memiliki titik berderajat $1$, tidak ada dua titik dengan himpunan \textit{open-neighborhood} yang sama, dan tidak ada dua titik dengan himpunan \textit{close-neighborhood} yang sama. Artinya, $Z=V(G)$, dan kita akan melakukan proses pengumpulan semua subhimpunan $V(G)$, dan memeriksa satu-per-satu apakah subhimpunan tersebut merupakan himpunan lokasi sentroidal dari $G$.
	\item Jika $B$ bukan merupakan himpunan kosong, maka perlu diperiksa apakah $B$ merupakan himpunan lokasi sentroidal. Jika ya, maka kita selesai. Jika tidak, kita akan melakukan proses pengumpulan semua subhimpunan dari $Z$, dimulai dari kardinalitas yang terkecil. Untuk setiap subhimpunan $Z^*$ dari himpunan $Z$, kita periksa apakah $B^*=B\cup Z^*$ merupakan himpunan lokasi sentroidal bagi $G$.
\end{enumerate}
\vspace{-\topsep}
\begin{algorithm}
	\floatname{algorithm}{Algoritma}
	\caption{Algoritma Basis Sentroidal}
	\begin{algorithmic}[1]
		\Procedure{ABS}{$B$,$Z$,$D$}
		\If {$|B|>0$}
		\If{$B$ adalah himpunan lokasi sentroidal}
		\State dimensi $\gets |B|$
		\State basis $\gets B$
		\Else
		\State [basis, dimensi] $\gets$ \textbf{compute} $\text{sentroidal}(B,Z,D)$
		\EndIf
		\Else	
		\State [basis, dimensi] $\gets$ \textbf{compute} $\text{sentroidal}(B,Z,D)$
		\EndIf
		\EndProcedure
	\end{algorithmic}
\end{algorithm}
Fungsi $\text{sentroidal}(B,Z,D)$ adalah fungsi untuk mencari basis dan dimensi dari matriks jarak $D$, himpunan $B$, dan $Z=V(G)\backslash B$ seperti yang telah kita cari sebelumnya.
\begin{algorithm}
	\floatname{algorithm}{Algoritma}
	\caption{Algoritma Sentroidal}
	\begin{algorithmic}[1]
		\Procedure{sentroidal}{$B$,$Z$,$D$}
		\State $i\gets 1$
		\While{$i<|Z|$}
		\State $SS=$ Kumpulan semua subhimpunan dari $Z$ dengan kardinalitas $i$
		\For {$m\gets1,|SS|$}
		\State $b\gets B\cup SS(m)$\Comment{$SS(m)$ subhimpunan urutan ke-$m$}
		\State $s \gets$ "b merupakan himpunan lokasi sentroidal"
		\If{$s=true$}
		\State dim $\gets|b|$
		\State \textbf{break}
		\EndIf
		\EndFor
		\If{$s=true$}
		\State \textbf{break}
		\Else
		\State $i\gets i+1$
		\EndIf
		\EndWhile
		\State basis $\gets$ b
		\State dimensi $\gets$ dim
		\EndProcedure
	\end{algorithmic}
\end{algorithm}

\newpage
\section{Hasil Algoritma}
Dalam subbab ini kita akan menggunakan algoritma di atas untuk menentukan dimensi sentroidal dari beberapa graf, seperti untuk graf yang telah dibahas pada bab sebelumnya, dan juga graf yang telah dikenal seperti graf lintasan, graf siklus, graf roda, dan graf kipas untuk orde yang kecil. Algoritma di atas juga digunakan untuk mencari graf sirkulan dengan generator kecil. Semua hasil algoritma disajikan dalam bentuk tabel-tabel berikut.

\begin{table}[h!]\label{tabel1}
	\centering
	\begin{tabular}{|c|c|c|c|c|c|}
		\hline
		$n$&$CD(H_n)$&$CD(n-\text{matahari})$&$CD(n-\text{sisir})$&$CD(n-\text{mahkota})$&$CD(K_n\boxdot P_2)$\\
		\hline
		\textbf{$2$}&$3$&$3$&$2$&$-$&$3$\\
		\textbf{$3$}&$4$&$4$&$4$&$3$&$3$\\
		\textbf{$4$}&$5$&$6$&$6$&$4$&$4$\\
		\textbf{$5$}&$6$&$7$&$7$&$5$&$4$\\
		\textbf{$6$}&$7$&$8$&$8$&$6$&$5$\\
		\textbf{$7$}&$8$&$10$&$10$&$6$&$6$\\
		\textbf{$8$}&$9$&$11$&$11$&$7$&$7$\\
		\textbf{$9$}&$10$&$12$&$12$&$8$&$8$\\
		\textbf{$10$}&$11$&$14$&$14$&$9$&$9$\\
		%\textbf{$11$}&$12$&$15$&$15$&$10$&$10$\\
		\hline
	\end{tabular}
	\caption{Dimensi Sentroidal graf $H_n$, graf $n-$matahari, graf $n-$sisir, graf $n-$mahkota, dan graf $K_n\boxdot P_2$.}
\end{table}

\begin{table}[h!]
	\centering
	\begin{tabular}{|c|c|c|c|c|}
		\hline
		$m$&$n$&$CD(T)$&$CD(B(m,n))$&$CD(D_n^{(m)})$\\
		\hline 
		2&2&2&3&2\\
		3&2&3&4&3\\
		4&2&4&5&4\\
		5&2&5&6&5\\
		6&2&6&7&6\\
		7&2&7&8&7\\
		2&3&4&4&3\\
		3&3&9&6&4\\
		4&3&16&8&5\\
		5&3&25&10&6\\
		2&4&8&5&4\\
		3&4&27&7&6\\
		4&4&64&9&8\\
		2&5&16&7&6\\
		3&5&81&10&9\\
		\hline
	\end{tabular}
\caption{Dimensi Sentroidal graf $T$ pohon $m-ary$ sempurna pada level $n$, graf pohon pisang $B(m,n)$, dan graf kincir angin $D_n^{(m)}$.}
\end{table}
$ $\\
\begin{table}[h!]
	\centering
	\begin{tabular}{|c|c|c|c|c|}
		\hline
		$n$&$CD(P_n)$&$CD(C_n)$&$CD(W_n)$&$CD(F_n)$\\
		\hline
		3&2&3&2&2\\
		4&3&3&3&3\\
		5&3&3&3&3\\
		6&4&3&3&3\\
		7&4&3&3&4\\
		8&4&3&4&4\\
		9&4&3&4&4\\
		10&5&3&5&5\\
		11&5&4&5&5\\
		12&5&3&6&6\\
		13&5&4&6&6\\
		14&6&4&7&7\\
		15&6&4&7&7\\
		16&6&4&7&7\\
		17&6&4&8&8\\
		18&6&4&8&8\\
		19&6&5&9&9\\
		20&7&4&9&9\\
		\hline
	\end{tabular}
	\caption{Dimensi sentroidal pada graf $P_n$, graf $C_n$, graf $W_n$, dan graf $F_n$.}
\end{table}

\begin{table}[h!]
	\centering
	\begin{tabular}{|c|c|c|c|c|}
		\hline
		$n$&$CD(C_n(1,2))$&$CD(C_n(1,3))$&$CD(C_n(2,3))$&$CD(C_n(1,2,3))$\\
		\hline
		6&3&4&3&5\\
		7&3&3&3&6\\
		8&3&6&3&4\\
		9&4&4&4&4\\
		10&3&5&5&4\\
		11&4&4&4&4\\
		12&3&5&4&4\\
		13&4&4&4&5\\
		14&4&5&4&4\\
		15&4&4&4&4\\
		16&4&5&4&4\\
		17&4&4&4&4\\
		18&4&4&4&4\\
		19&4&4&4&5\\
		20&4&5&4&4\\
		21&4&4&4&5\\
		22&4&5&4&4\\
		\hline
	\end{tabular}
	\caption{Dimensi sentroidal graf $C_n(1,2)$, $C_n(1,3)$, $C_n(2,3)$, dan $C_n(1,2,3)$.}
\end{table}
\FloatBarrier
Untuk $n=2$, graf $n-$mahkota merupakan graf tak terhubung, kemudian $K_2\boxdot P_2\cong C_4$, sehingga $CD(K_2\boxdot P_2)=CD(C_4)=3$. Lalu, untuk graf kincir angin $D_n^{(m)}$ isomorfik dengan graf bipartit lengkap $K_{1,m}$, sehingga $CD(D_2^{(m)})=CD(K_{1,m})=m$.



